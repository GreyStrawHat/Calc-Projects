\documentclass[letterpaper,12pt]{article}
\usepackage{tabularx} % extra features for tabular environment
\usepackage{amsmath}  % improve math presentation
\usepackage{graphicx} % takes care of graphic including machinery
\usepackage[margin=1in,letterpaper]{geometry} % decreases margins
\usepackage{cite} % takes care of citations
\usepackage[final]{hyperref} % adds hyper links inside the generated pdf file
\hypersetup{
	colorlinks=true,       % false: boxed links; true: colored links
	linkcolor=blue,        % color of internal links
	citecolor=blue,        % color of links to bibliography
	filecolor=magenta,     % color of file links
	urlcolor=blue         
}
\usepackage{blindtext}
\usepackage{dirtree}
\usepackage{listings}
\usepackage{color}

\definecolor{dkgreen}{rgb}{0,0.6,0}
\definecolor{gray}{rgb}{0.5,0.5,0.5}
\definecolor{mauve}{rgb}{0.58,0,0.82}

	\lstset{frame=tb,
	language=Java,
	aboveskip=3mm,
	belowskip=3mm,
	showstringspaces=false,
	columns=flexible,
	basicstyle={\small\ttfamily},
	numbers=none,
	numberstyle=\tiny\color{gray},
	keywordstyle=\color{blue},
	commentstyle=\color{dkgreen},
	stringstyle=\color{mauve},
	breaklines=true,
	breakatwhitespace=true,
	tabsize=3
}

%++++++++++++++++++++++++++++++++++++++++


\begin{document}
	
	\title{Simple Calc}
	\author{Cyber Solutions Development - Georgia}
	\date{\today}
	\maketitle
	
	\begin{abstract}
		Your task is build a simple calculator application that will take an equation as an argument, and produce the result to standard out.
	\end{abstract}
	
	
	\section{Requirements}
	
	In this assignment, you will build a simple calculator application that will compute simple equations and produce the result. The requirements are below:
	
	\subsection{Basic Requirements}
		\begin{enumerate}
			\item Written in C
			\item Take a single argument that is an equation of the form (operand1) (operator) (operand2)
			\item Handle bad inputs such as bad format as described above, divide by zero, or interger overflow
			\item Single binary with the usage statement ./simplecalc <equation>
		\end{enumerate}
	
	\subsection{Specific Requirements}
	\subsubsection{Required Operators}
		\begin{enumerate}
			\item Addition 
			\item Subtraction
			\item Multiplication
			\item Division
			\item Modulo 
			\item Left shift 
			\item Right shift 
			\item And
			\item Or
			\item XOR
			\item Rotate Left
			\item Rotate Right
		\end{enumerate}
	
	\section{Deliverables}
	Your code should have the following file structure:
	\dirtree{%
	.1 SimpleCalc.
	.2 src.
	.3 source-files.
	.2 hdrs.
	.3 header-files.
	.2 docs.
	.3 documentation.
	.2 CMakeLists.txt.
	}\hfill

	Your code should build and compile with the following shell script ran from the SimpleCalc directory:

	\begin{lstlisting}
	// build.sh
	mkdir build
	cd build
	cmake ..
	\end{lstlisting}
	
	
	\section{Notes to grader}
	The purpose of this assignment is NOT to write fancy C code. Use this assignment to achieve the following objectives:
	\begin{enumerate}
		\item Hammer down on the coding standard. By the end of the assignment your mentee should understand the coding standard.
		\item Code organization. Ensure your mentee does not implement the solving functionality in main(). Encourage them to implement separate functions for each operator. This will help them throughout the next assignments.
		\item Building projects with CMake. CMake is a powerful build system, but requires a learning curve. Use this project to introduce them to CMake.
		\item Git tradecraft. Have your mentee branch off of devel, commit their work, submit a merge request to devel, etc. Make devel a protected branch.
	\end{enumerate}

	\section{JQR Sections Covered}
	\begin{itemize}
		\item 3.1.3 (all)
		\item 3.1.5 (all)
		\item 3.1.6 (a, b)
		\item 3.1.8 (a, c, d, e) 
		\item 3.1.9 
	\end{itemize}
	
	
\end{document}
